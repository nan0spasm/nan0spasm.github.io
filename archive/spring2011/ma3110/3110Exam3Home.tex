\documentclass[11pt]{article}

\usepackage{url}
\usepackage{tikz}
\usepackage{fancyhdr}
\usepackage[margin=.7in]{geometry}
\usepackage[hang,flushmargin,symbol*]{footmisc}
\usepackage{amsmath}
\usepackage{todonotes}
\usepackage{amsthm}
\usepackage{amssymb}
\usepackage{mathtools}
\usepackage{enumitem}
\usepackage{graphicx}
\usepackage{color}
\usepackage{tipa} %to get \textpipe to work
\definecolor{darkblue}{rgb}{0, 0, .6}
\definecolor{grey}{rgb}{.7, .7, .7}
\usepackage[breaklinks]{hyperref}
\hypersetup{
	colorlinks=true,
	linkcolor=darkblue,
	anchorcolor=darkblue,
	citecolor=darkblue,
	pagecolor=darkblue,
	urlcolor=darkblue,
	pdftitle={},
	pdfauthor={}
}

\theoremstyle{definition} 
\newtheorem{theorem}{Theorem}
\newtheorem{lemma}[theorem]{Lemma}
\newtheorem{claim}[theorem]{Claim}
\newtheorem{corollary}[theorem]{Corollary}
\newtheorem{conjecture}[theorem]{Conjecture}
\newtheorem{definition}[theorem]{Definition}
\newtheorem{example}[theorem]{Example}
\newtheorem{remark}[theorem]{Remark}
\newtheorem{important}[theorem]{Important Note}
\newtheorem{recall}[theorem]{Recall}
\newtheorem{note}[theorem]{Note}
\newtheorem{question}[theorem]{Question}

\newcommand{\blank}{\underline{\ \ \ \ \ \ \ \ \ \ \ \ \ \ \ \ \ \ \ }}
\newcommand{\ds}{\displaystyle}

%todo commants
\newcommand{\insertref}[1]{\todo[color=green!40]{#1}}
\newcommand{\comment}[1]{\todo[color=blue!20!white,inline]{#1}}
\setlength{\marginparwidth}{2cm}

\setlength{\parindent}{0pt}
\setlength{\fboxsep}{10pt}

%%%%%%Header/Footer%%%%%%%

\pagestyle{fancy}

\lhead{\scriptsize  MA3110: Logic, Proof, \& Axiomatic Systems - Spring 2011} 
\chead{} 
\rhead{\scriptsize Take-Home Portion of Final Exam} 
\lfoot{\scriptsize This work is licensed under the \href{http://creativecommons.org/licenses/by-sa/3.0/us/}{Creative Commons Attribution-Share Alike 3.0 License}.} 
\cfoot{} 
\rfoot{\scriptsize Written by \href{http://oz.plymouth.edu/~dcernst}{D.C. Ernst}} 
\renewcommand{\headrulewidth}{0.4pt} 
\renewcommand{\footrulewidth}{0.4pt} 

%%%%%%%%%%%%%%%%%%%

\begin{document}

\begin{center}

{\Large\bf MA3110: Logic, Proof, \& Axiomatic Systems - Spring 2011}\\
\smallskip
{\Large\bf Take-Home Portion of Final Exam}

\bigskip

  \fbox{\parbox{7in}{
    \vspace{12pt}
    \textbf{\large NAME:}
    \vspace{12pt}
  }}

\end{center}

\setlength{\fboxsep}{10pt}

\section*{Instructions}

This portion of Final Exam is worth 20 points.  Prove any \textbf{two} theorems on the following page.  Each proof is worth 10 points.

\bigskip

You should write in \emph{complete sentences}.  I expect your proofs to be \emph{well-written, neat}, and \emph{organized}.  Do \emph{not} turn in rough drafts.  What you turn in should be the ``polished'' version of potentially several drafts.  Feel free to type up your final version.  

\bigskip

The \LaTeX\ source file of this exam is also available if you are interested in typing up your solutions using \LaTeX.  I'll help you do this if you'd like.

\bigskip

The simple rules for the exam are:

\begin{enumerate}
\item You may freely use any theorems that we have discussed in class, but you should make it clear where you are using a previous result and which result you are using.  For example, if a sentence in your proof follows from Theorem 1.41, then you should say so.
\item Unless you prove them, you cannot use any results from the course notes or otherwise that we have not covered.
\item You are NOT allowed to copy someone else's work.
\item You are NOT allowed to let someone else copy your work.
\item You are allowed to discuss the problems with each other and critique each other's work.
\end{enumerate}

This portion of the Final Exam is due by 5\textsc{pm} on \textbf{Friday, May 20}.  You should turn in this cover page and all of the work that you have decided to submit.

\bigskip

To convince me that you have read and understand the instructions, sign in the box below.

\bigskip

  \fbox{\parbox{7in}{
    \vspace{12pt}
    \textbf{\large Signature:} \hfill
    \vspace{12pt}
  }}

\bigskip

Good luck and have fun!

\newpage

\begin{theorem}
Let $f:X\to Y$ be a one-to-one correspondence.  For $A\subseteq X$, define $f(A)=\{f(a):a\in A\}$.  Suppose that $\Omega$ is a partition of $X$.  Then
\[
\Omega^{*}=\{f(A):A\in\Omega\}
\]
is a partition of $Y$.
\end{theorem}

\begin{theorem}
Let $f:X\to Y$ be a function.  Define a relation on $X$ via $x_{1}\sim x_{2}$ iff $f(x_{1})=f(x_{2})$.  Then $\sim$ is an equivalence relation.
\end{theorem}

\begin{theorem}
Let $f:X\to Y$ and $g:Y\to Z$ be functions.  If $f$ and $g$ are both one-to-one correspondences, then $(g\circ f)^{-1}=f^{-1}\circ g^{-1}$.\footnote{This is Theorem 3.107 from the course notes.}
\end{theorem}

\begin{theorem}
For all $n\in\mathbb{N}$, $\displaystyle \frac{1}{1\cdot 2}+\frac{1}{2\cdot 3}+\cdots +\frac{1}{n(n+1)}=\frac{n}{n+1}$.
\end{theorem}

\end{document}
