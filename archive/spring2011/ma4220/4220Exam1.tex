\documentclass[11pt]{article}

\usepackage{url}
\usepackage{tikz}
\usepackage{fancyhdr}
\usepackage[margin=.7in]{geometry}
\usepackage[hang,flushmargin,symbol*]{footmisc}
\usepackage{amsmath}
\usepackage{todonotes}
\usepackage{amsthm}
\usepackage{amssymb}
\usepackage{mathtools}
\usepackage{enumitem}
\usepackage{graphicx}
\usepackage{color}
\usepackage{tipa} %to get \textpipe to work
\definecolor{darkblue}{rgb}{0, 0, .6}
\definecolor{grey}{rgb}{.7, .7, .7}
\usepackage[breaklinks]{hyperref}
\hypersetup{
	colorlinks=true,
	linkcolor=darkblue,
	anchorcolor=darkblue,
	citecolor=darkblue,
	pagecolor=darkblue,
	urlcolor=darkblue,
	pdftitle={},
	pdfauthor={}
}

\theoremstyle{definition} 
\newtheorem{theorem}{Theorem}
\newtheorem{lemma}[theorem]{Lemma}
\newtheorem{claim}[theorem]{Claim}
\newtheorem{corollary}[theorem]{Corollary}
\newtheorem{conjecture}[theorem]{Conjecture}
\newtheorem{definition}[theorem]{Definition}
\newtheorem{example}[theorem]{Example}
\newtheorem{remark}[theorem]{Remark}
\newtheorem{important}[theorem]{Important Note}
\newtheorem{recall}[theorem]{Recall}
\newtheorem{note}[theorem]{Note}
\newtheorem{question}[theorem]{Question}

\newcommand{\blank}{\underline{\ \ \ \ \ \ \ \ \ \ \ \ \ \ \ \ \ \ \ }}
\newcommand{\ds}{\displaystyle}

%todo commants
\newcommand{\insertref}[1]{\todo[color=green!40]{#1}}
\newcommand{\comment}[1]{\todo[color=blue!20!white,inline]{#1}}
\setlength{\marginparwidth}{2cm}

\setlength{\parindent}{0pt}
\setlength{\fboxsep}{10pt}

%%%%%%Header/Footer%%%%%%%

\pagestyle{fancy}

\lhead{\scriptsize  MA4220: Number Theory- Spring 2011} 
\chead{} 
\rhead{\scriptsize Exam 1} 
\lfoot{\scriptsize This work is licensed under the \href{http://creativecommons.org/licenses/by-sa/3.0/us/}{Creative Commons Attribution-Share Alike 3.0 License}.} 
\cfoot{} 
\rfoot{\scriptsize Written by \href{http://oz.plymouth.edu/~dcernst}{D.C. Ernst}} 
\renewcommand{\headrulewidth}{0.4pt} 
\renewcommand{\footrulewidth}{0.4pt} 

%%%%%%%%%%%%%%%%%%%

\begin{document}

\begin{center}

{\Large\bf MA4220: Number Theory - Spring 2011}\\
\smallskip
{\Large\bf Exam 1}

\bigskip

  \fbox{\parbox{7in}{
    \vspace{12pt}
    \textbf{\large NAME:}
    \vspace{12pt}
  }}

\end{center}

\setlength{\fboxsep}{10pt}

\section*{Instructions}

This exam is worth 15\% of your overall grade and all of the problems have equal weight.  For each part of the exam, read the instructions carefully.

\bigskip

I expect your proofs to be \emph{well-written, neat, and organized}.  You should write in \emph{complete sentences}.  Do not turn in rough drafts.  What you turn in should be the ``polished'' version of potentially several drafts.  Feel free to type up your final version.  

\bigskip

The \LaTeX\ source file of this exam is also available if you are interested in typing up your solutions using \LaTeX.  I'll help you do this if you'd like.

\bigskip

The simple rules for the exam are:

\begin{enumerate}
\item You may freely use any theorems that we have discussed in class, but you should make it clear where you are using a previous result and which result you are using.  For example, if a sentence in your proof follows from Theorem 1.41, then you should say so.
\item Unless you prove them, you cannot use any results from the textbook that we have not covered.
\item You are NOT allowed to consult external sources when working on the exam.  This includes people outside of the class, other textbooks, and online resources.
\item You are NOT allowed to copy someone else's work.
\item You are NOT allowed to let someone else copy your work.
\item You are allowed to discuss the problems with each other and critique each other's work.
\end{enumerate}

The exam is due to my office by 5\textsc{pm} on \textbf{Friday, March 11}.  You should turn in this cover page and all of the work that you have decided to submit.

\bigskip

To convince me that you have read and understand the instructions, sign in the box below.

\bigskip

  \fbox{\parbox{7in}{
    \vspace{12pt}
    \textbf{\large Signature:} \hfill
    \vspace{12pt}
  }}

\bigskip

Good luck and have fun!

\newpage

\section*{Part 1}
Complete \emph{one} of the following problems.  You need to justify your answers with sufficient work and should not rely on technology.

\begin{enumerate}

\item Exercise 1.50 on page 21.

\item Exercise 1.54 on page 22.

\end{enumerate}

\section*{Part 2}

Complete any 3 of the following problems.

\begin{enumerate}[resume]

\item The Fibonacci sequence is an infinite sequence that begins $1,1,2,3, 5,8,13,\ldots$.  Let $F_{n}$ denote the $n$th Fibonacci number.  The sequence is defined by $F_{1}=1$, $F_{2}=1$, and $F_{n}=F_{n-1}+F_{n-2}$ for all $n\geq 3$.  Prove that for any natural number $n$, $(F_{n},F_{n+1})=1$.

\item Suppose $p$ and $q$ are prime numbers such that $q=p+2$.  (Such primes are called \emph{twin primes}.)  Also, suppose that $p>3$.  Prove that $p\equiv 2 \mbox{ (mod } 3)$.

\item Theorem 2.28 on page 34.

\item Which of the following statements is true?  Prove your assertion.
\begin{enumerate}

\item There are no integers $n$ such that $n^{2}-1$ is prime.

\item There are a nonzero but finite number of integers $n$ such that $n^{2}-1$ is prime.

\item There are infinitely many integers $n$ such that $n^{2}-1$ is prime.

\end{enumerate}

\item Prove that if $n+1$ integers are chosen from the set $\{1,2,...,2n\}$ then at least $2$ of the chosen integers are relatively prime.\footnote{Make a clever use of the Fundamental Theorem of Arithmetic.}

\end{enumerate}

\section*{Part 3}

Here are the instructions for this portion of the exam.

\begin{itemize}

\item Prove both of the following theorems.

\item You are required to type your proofs and submit them to me via email at \url{dcernst@plymouth.edu}.  

\item Please put each proof on its own page.  If you choose to type your entire exam, I would like these problems to be in a separate file.  

\item Send me a PDF file and name your file according to: \texttt{Exam1Part3Last-Name.pdf}.  

\item Do \emph{not} include your name anywhere on the typeset document (but you should include your last name in the filename).

\end{itemize}

These proofs will be sent to students in a number theory course at Wellesley College to be peer reviewed.  You will receive a critical review of your proof from a student at Wellesley, but their critique will \emph{not} impact your grade.  Similarly, we will be reviewing proofs submitted by students from Wellesley in the near future.

\begin{theorem}
If $a,b\in\mathbb{Z}$, not both 0, and $k\in\mathbb{N}$, then $\gcd(ka,kb)=k\cdot \gcd(a,b)$.\footnote{This is Theorem 1.55 on page 22.}
\end{theorem}

\begin{theorem}
Suppose that $p_{1}, p_{2}, \ldots, p_{n}$ are distinct primes (i.e., each $p_{i}$ is prime, but $p_{i}\neq p_{j}$ for $i\neq j$).  Then $\sqrt{p_{1}p_{2}\cdots p_{n}}$ is irrational.
\end{theorem}



\end{document}
