\documentclass[11pt]{article}

\usepackage{url}
\usepackage{tikz}
\usepackage{fancyhdr}
\usepackage[margin=.7in]{geometry}
\usepackage[hang,flushmargin,symbol*]{footmisc}
\usepackage{amsmath}
\usepackage{todonotes}
\usepackage{amsthm}
\usepackage{amssymb}
\usepackage{mathtools}
\usepackage{enumitem}
\usepackage{graphicx}
\usepackage{color}
\usepackage{tipa} %to get \textpipe to work
\definecolor{darkblue}{rgb}{0, 0, .6}
\definecolor{grey}{rgb}{.7, .7, .7}
\usepackage[breaklinks]{hyperref}
\hypersetup{
	colorlinks=true,
	linkcolor=darkblue,
	anchorcolor=darkblue,
	citecolor=darkblue,
	pagecolor=darkblue,
	urlcolor=darkblue,
	pdftitle={},
	pdfauthor={}
}

\theoremstyle{definition} 
\newtheorem{theorem}{Theorem}
\newtheorem{lemma}[theorem]{Lemma}
\newtheorem{claim}[theorem]{Claim}
\newtheorem{corollary}[theorem]{Corollary}
\newtheorem{conjecture}[theorem]{Conjecture}
\newtheorem{definition}[theorem]{Definition}
\newtheorem{example}[theorem]{Example}
\newtheorem{remark}[theorem]{Remark}
\newtheorem{important}[theorem]{Important Note}
\newtheorem{recall}[theorem]{Recall}
\newtheorem{note}[theorem]{Note}
\newtheorem{question}[theorem]{Question}

\newcommand{\blank}{\underline{\ \ \ \ \ \ \ \ \ \ \ \ \ \ \ \ \ \ \ }}
\newcommand{\ds}{\displaystyle}

%todo commants
\newcommand{\insertref}[1]{\todo[color=green!40]{#1}}
\newcommand{\comment}[1]{\todo[color=blue!20!white,inline]{#1}}
\setlength{\marginparwidth}{2cm}

\setlength{\parindent}{0pt}
\setlength{\fboxsep}{10pt}

\DeclareMathOperator{\ord}{ord}

%%%%%%Header/Footer%%%%%%%

\pagestyle{fancy}

\lhead{\scriptsize  MA4220: Number Theory - Spring 2011} 
\chead{} 
\rhead{\scriptsize Final Exam} 
\lfoot{\scriptsize This work is licensed under the \href{http://creativecommons.org/licenses/by-sa/3.0/us/}{Creative Commons Attribution-Share Alike 3.0 License}.} 
\cfoot{} 
\rfoot{\scriptsize Written by \href{http://oz.plymouth.edu/~dcernst}{D.C. Ernst}} 
\renewcommand{\headrulewidth}{0.4pt} 
\renewcommand{\footrulewidth}{0.4pt} 

%%%%%%%%%%%%%%%%%%%

\begin{document}

\begin{center}

{\Large\bf MA4220: Number Theory - Spring 2011}\\
\smallskip
{\Large\bf Final Exam}

\bigskip

  \fbox{\parbox{7in}{
    \vspace{12pt}
    \textbf{\large NAME:}
    \vspace{12pt}
  }}

\end{center}

\setlength{\fboxsep}{10pt}

\section*{Instructions}

The Final Exam is worth 15\% of your overall grade.  For each part of the exam, read the instructions carefully.

\bigskip

I expect your proofs to be \emph{well-written, neat, and organized}.  You should write in \emph{complete sentences}.  Do not turn in rough drafts.  What you turn in should be the ``polished'' version of potentially several drafts.  Feel free to type up your final version.  

\bigskip

The \LaTeX\ source file of this exam is also available if you are interested in typing up your solutions using \LaTeX.  I'll help you do this if you'd like.

\bigskip

The simple rules for the exam are:

\begin{enumerate}
\item You may freely use any theorems that we have discussed in class, but you should make it clear where you are using a previous result and which result you are using.  For example, if a sentence in your proof follows from Theorem 1.41, then you should say so.
\item Unless you prove them, you cannot use any results from the textbook that we have not covered.
\item You are NOT allowed to consult external sources when working on the exam.  This includes people outside of the class, other textbooks, and online resources.
\item You are NOT allowed to copy someone else's work.
\item You are NOT allowed to let someone else copy your work.
\item You are allowed to discuss the problems with each other and critique each other's work.
\end{enumerate}

The Final Exam is due to my office by 5\textsc{pm} on \textbf{Friday, May 20}.  You should turn in this cover page and all of the work that you have decided to submit.

\bigskip

To convince me that you have read and understand the instructions, sign in the box below.

\bigskip

  \fbox{\parbox{7in}{
    \vspace{12pt}
    \textbf{\large Signature:} \hfill
    \vspace{12pt}
  }}

\bigskip

Good luck and have fun!

\newpage

\section*{Part 1}

Suppose that $pq=11537$ and $E=85$ and consider the following message that has been encrypted using $pq$ and $E$:
\begin{center}
8664, 1840,2615, 178, 1, 5483, 10096, 2373, 2615, 11393, 178, 8199, 10096, 178, 1793, 10098, 11393, 8664, 8321, 8664, 2373, 10098
\end{center}
Assume that the block length is 2 and that the dictionary from the RSA handout was used to encrypt.  Use whatever means necessary to decipher the message.  You may rely on technology such as \texttt{Sage} or \texttt{WolframAlpha}.  However, you should explain what steps you took to solve the problem and how you utilized the technology.  

\section*{Part 2}

Complete any \textbf{four} of the following theorems.

\begin{theorem}[Chinese Remainder Theorem]
Suppose $n_{1}, n_{2},\ldots, n_{L}$ are positive integers that are pairwise relatively prime, that is, $(n_{i},n_{j})=1$ for $i\neq j$, $1\leq i,j,\leq L$.  Then the system of $L$ congruences
\begin{align*}
x & \equiv a_{1} \pmod{n_{1}}\\
x & \equiv a_{2} \pmod{n_{2}}\\
& \vdots \\
x & \equiv  a_{L} \pmod{n_{L}}
\end{align*}
has a unique solution modulo the product $n_{1}n_{2}\cdots n_{L}$.\footnote{This is Theorem 3.29 in our textbook.  We discussed a potential proof for this one day in class, but no one presented a satisfactory proof.  So, now is your chance.  You can only use theorems that come before this theorem in the book to prove it.}
\end{theorem}

\begin{theorem}[Euler's Theorem]
If $a$ and $n$ are integers with $n>0$ and $(a,n)=1$, then
\[
a^{\phi(n)}\equiv 1 \pmod{n}.\footnote{This is Theorem 4.32 in our textbook.  As with the Chinese Remainder Theorem, no one presented a satisfactory proof.  You can only use theorems that come before this theorem in the book to prove it.}
\]
\end{theorem}

\begin{theorem}
Let $n$ and $m$ be natural numbers that are relatively prime, and let $a$ be an integer.  If $x\equiv a \pmod n$ and $x\equiv a\pmod m$, then $x\equiv a \pmod{nm}$.\footnote{This is Theorem 4.21 in the textbook. We've used this theorem a few times, but I'm pretty sure that we never proved it.}
\end{theorem}

\begin{theorem}
Suppose that $p$ is a prime number and $a$ is an integer such that $(a,p)=1$.  Then for all $k\in\mathbb{N}$, we have
\[
\ord_{p}(a^{k})=\frac{\ord_{p}(a)}{(k,\ord_{p}(a))}.
\]
\end{theorem}

\begin{theorem}
Suppose that $n$ is a natural number and $a$ is an integer such that $(a,n)=1$ and $a^{n-1}\equiv 1 \pmod n$.  Then $\ord_{n}(a)<n-1$ iff there exists a prime $p$ such that $p\mid n-1$ and
\[
a^{\frac{n-1}{p}}\equiv 1 \pmod n.
\]
\end{theorem}

\begin{theorem}
Suppose that $n$ is a natural number and $a$ is an integer such that $(a,n)=1$ and $a^{n-1}\equiv 1 \pmod n$.  If $\ord_{n}(a)=n-1$, then $n$ is prime.
\end{theorem}

\end{document}