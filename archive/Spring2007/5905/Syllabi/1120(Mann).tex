% Course syllabus for Math 1120-003 Spring 2006
% LaTeX by Allen Mann

\documentclass[11pt]{article}
\usepackage{graphicx}
\usepackage{amssymb}

\textwidth = 6.5 in
\textheight = 9.5 in
\oddsidemargin = 0.0 in
\evensidemargin = 0.0 in
\topmargin = 0.0 in
\headheight = -0.25 in
\headsep = 0.0 in
\parskip = 0.2in
\parindent = 0.0in

%Allen's Macro's
\newcommand{\AM}{\textsc{a.m.}}
\newcommand{\PM}{\textsc{p.m.}}
\newcommand{\book}[1]{\textit{#1}} 
\newcommand{\url}[1]{\texttt{#1}} 

\begin{document}

\begin{center}
\framebox{\parbox{4 in}{
\begin{center}
\textbf{\large MATH 1120-003} \\[6 pt]
\textbf{\large Spirit \& Uses of Mathematics 2} \\[10 pt]
\textbf{Spring 2006}
\end{center}
}}
\end{center}


\begin{tabbing}
\textit{Office Hours:} \quad   \= Allen.Mann@Colorado.EDU   \kill
\textit{Instructor:}		\> Allen Mann \\
\textit{Office:}			\> Mathematics 362 \\
\textit{E-mail:}			\> \url{Allen.Mann@Colorado.EDU} \\
\textit{Web:}			\> \url{http://math.colorado.edu/\~\,$\!$almann/math1120}
\end{tabbing}

\vspace{-0.2 in}

\begin{description}

\item[Lecture:] MWF 2:00--2:50 \PM\ in Duane G2B41.

\item[Office Hours:] MWF 1:00-1:50 \PM\ in Mathematics 362 or by appointment.

\item[Textbook: ]
\book{A Problem Solving Approach to Mathematics for Elementary School Teachers} (8th edition), by R. Billstein, S. Libeskin, and J. Lott. Pearson-Addison-Wesley, 2004.

\item[Required Materials:] Straightedge, compass, protractor, graph paper.

\item[Course Description:]
This course is a continuation of MATH 1110, and is designed to prepare the student to teach mathematics  at the elementary-school level. It is a math class, but it is probably unlike other math classes you have taken. Our aim is to develop your critical-thinking and analytic skills by studying elementary-school mathematics from an advanced point of view. Since some day you will be expected to \emph{teach} mathematics, rather than simply \emph{do} mathematics, a deeper understanding of the material will be expected. Topics we will cover this semester include set theory, probability, statistics, and geometry.

\item[Grading: ]
\begin{tabbing}
\hspace{0.5 in}    \= Problems of the Week \quad  \=    \kill
\> Homework  \> 10\%    \\
\> Quizzes  \> 10\%    \\
\> Problems of the Week  \> 10\%    \\
\> Midterm Exams (3)  \> 45\% \\
\> Final Exam \> 25\% \\ 
\end{tabbing}

\vspace{-0.18 in}

\item[Homework \& Quizzes: ]
Daily homework assignments will be graded on completion only. Homework will be due at the beginning of the following class. Students are encouraged to work together on homework. Weekly quizzes will be graded for correctness. Quizzes will usually be on Wednesday.

\item[Problems of the Week: ]
Problems of the Week will be assigned on Monday and due on Friday. Written solutions to the POWs will be graded based on two criteria: presentation of the method used to solve the problem and the correctness of that method. In addition, every student will be required present their solution in front of the class at least once during the semester. If you are having trouble solving a POW, you should first ask the instructor for assistance. If that doesn't work, you may consult with your fellow students. You are not allowed to use the Internet.

\item[Calculators: ] Calculators will not be allowed on quizzes or exams.

\item[Midterm Exams: ]
February 17, March 24, and April 28.

\item[Final Exam: ]
Saturday, May 6, 1:30--4:00 \PM

\pagebreak

\item[Students with Disabilities: ]
Students with disabilities who qualify for academic accommodations must provide a letter from Disability Services and discuss specific needs with me, preferably during the first two weeks of class.  Disability Services determines accommodations based on documented disabilities. Disability Services in located in Willard 322, (303) 492-8671, \\ \url{http://www.colorado.edu/sacs/disabilityservices}.

\item[Religious Obligations: ]
Students who have a religious obligation that conflicts with one of the scheduled exams, assignments, or other required attendance should notify me at least two weeks in advance of the conflict to request special accomodation. Please refer to the University's policy on religious obligations at \url{http://www.colorado.edu/policies}.

\item[Classroom Behavior: ]
See \url{http://www.colorado.edu/policies/classbehavior.html}.

\item[Honor Code: ] 
Details of the Student Honor Code system can be found at \\
\url{http://www.colorado.edu/academics/honorcode} \\
\url{http://www.colorado.edu/policies/honor.html} \\
\url{http://www.colorado.edu/policies/acadinteg.html}

\item[Cheating: ]
Cheating is defined as using unauthorized materials or receiving unauthorized assistance during an examination or other academic exercise. Examples of cheating include: copying the work of another student during an examination or other academic exercise (includes computer programming), or permitting another student to copy one�s work; taking an examination for another student or allowing another student to take one�s examination; possessing unauthorized notes, study sheets, examinations, or other materials during an examination or other academic exercise; collaborating with another student during an academic exercise without the instructor�s consent; and/or falsifying examination results.

\item[Plagiarism: ]
Plagiarism is defined as the use of another�s ideas or words without appropriate acknowledgment. Examples of plagiarism include: failing to use quotation marks when directly quoting from a source; failing to document distinctive ideas from a source; fabricating or inventing sources; and copying information from computer-based sources, i.e., the Internet.

\item[Unauthorized Possession or Disposition of Academic Materials: ]
Unauthorized possession or disposition of academic materials may include: selling or purchasing examinations, papers, reports or other academic work; taking another student�s academic work without permission; possessing examinations, papers, reports, or other assignments not released by an instructor; and/or submitting the same paper for multiple classes without advance instructor authorization and approval.

\end{description}

\end{document}