\documentclass[12pt]{article}
\usepackage{url}
\addtolength{\oddsidemargin}{-0.9in} 
\addtolength{\textwidth}{1.5in} 
\addtolength{\textheight}{2in} 
\addtolength{\topmargin}{-1.1in} 
\pagestyle{empty}



\begin{document}

\centerline{\large\bf Introduction to Linear Algebra, Fall 2006} 
\centerline{\large\bf MATH 3130, Section 001} 

\smallskip

\centerline{\bf MWF, 9:00--9:50 am, KTCH 303}

\bigskip

\noindent {\large{\bf Instructor Information:}}\\

\begin{tabular}{l}
\, {\bf  Instructor:}  Dana Ernst\\
\, {\bf  Office:} Math 214\\
\, {\bf  Office Hours:} MWF 10:00--11:00 am (or by appointment)\\
\, {\bf  Email:} Dana.Ernst@colorado.edu\\
\, {\bf  Web Page:}  \url{http://math.colorado.edu/~ ernstd}\\
\end{tabular}\\

\smallskip

\noindent {\large{\bf Course Information and Policies:}}

\begin{itemize}
  \item[] {\bf Description:} Examines basic properties of systems of linear equations, vector spaces, linear independence, dimension, linear transformations, matrices, determinants, eigenvalues, and eigenvectors.
  \item[] {\bf Prerequisites:} MATH 2400: Calculus III, or equivalent.
  \item[] {\bf Purpose:} The primary objective of this course is to aid students in becoming confident and competent in solving problems that require techniques developed in Linear Algebra.  Successful completion of MATH 3130 provides students with skills necessary for upper division mathematics, science, and computer science courses.  Students will have a working understanding of systems of linear equations, vector spaces, linear independence, dimension, linear transformations, matrices, determinants, eigenvalues, and eigenvectors.  Students will also improve their ability to read and write mathematics.  In particular, students will develop basic proof writing skills.  Also, the purpose of any mathematics class is to challenge and train the mind.  Learning mathematics enhances critical thinking and problem solving skills.
  \item[] {\bf Text:} {\it Linear Algebra and Its Applications}, 3rd Edition, by David C. Lay
  \item[] {\bf Attendance:}  Regular attendance is expected and is vital to success in this course.
  \item[] {\bf Homework:}  Homework will be assigned regularly (a list of problems will be coming soon) and collected at the beginning of class on the day it is due. You are allowed and encouraged to work together on homework.  However, each student is expected to turn in their own work, unless otherwise instructed.  Late homework will not be accepted without prior approval from me.  At least 3 of your lowest homework scores will be dropped at the end of the semester. 
  \item[] {\bf Exams:} There will be three midterm exams (Feb 9, Mar 9, Apr 13) and a final exam (Tuesday, May 8, 1:30--4:00 pm).  I will notify you at least a week ahead of time if the date of one of the midterms needs to be changed.  The final exam will be cumulative.  Make-up exams will only be given under extreme circumstances, as judged by me.  In general, it will be best to communicate conflicts ahead of time.
  \item[] {\bf Projects:}  There will be three group projects due throughout the semester (Feb 23, Mar 23, Apr 27). The emphasis of the projects will be proof writing.  Groups will consist of 2-3 students.  More information will be provided later.
  \item[] {\bf About Calculators:} Calculators will not be allowed on any of the exams (you won't need them anyway).  However, after the first few homework assignments you can use calculators and/or computer software when doing your homework:  WARNING:  Be sure that you know how to do calculations by hand come exam time!
   \item[] {\bf Other Comments:} Turn off your cell phones!!!  If you must be late for class, please try not to disrupt class.
\end{itemize}

\noindent {\large{\bf Course Evaluation:}}

\begin{itemize}
  \item[] {\bf Grading:} You will be graded on your written work, which will be judged on the basis of {\it correctness}, {\it completeness}, and {\it legibility}.
  \item[] {\bf Basis for Evaluation:} Your final grade will be determined by the scores of your homework, midterm exams, projects, and final exam.  To combine these items the following weights will be used:
  \begin{itemize}
  \item[] {\bf Homework:} 20\%
  \item[]  {\bf Projects:} 15\%
  \item[] {\bf Midterm Exams:} 15\% each (total of 45\%)
  \item[] {\bf Final exam:} 20\%
\end{itemize}

  \item[] {\bf Grade Determination:} Grades may be "massaged" at the end of the semester, but in general this is what you should expect:
\begin{itemize}
\item[] \begin{tabular}{@{}llllllll}
93-100\%   & A &&&&& 73-76\% & C \\
90-92\%   & A- &&&&& 70-72\% & C- \\
87-89\%   & B+ &&&&&  67-69\% & D+\\
83-86\% & B &&&&& 63-66\% & D \\
80-82\% & B- &&&&& 60-62\% & D- \\
77-79\% & C+ &&&&& 0-59\% & F \\
\end{tabular}
\end{itemize}
  
  \item[] There will be no extra credit assignments.

\end{itemize}

\noindent {\large{\bf Additional Information:}}

\begin{itemize}
  \item[] {\bf Important Dates:}
  \begin{itemize}
  \item[]
\begin{tabular}{@{}llllll}
Jan 31 (Wed): & 1st Drop Deadline &&&
   Mar 23 (Fri): & Project 2 Due\\
Feb 9 (Fri): & Exam 1 &&& 
  Mar 26-30  (Mon-Fri): & Spring Break\\
Feb 23 (Fri): & Project 1 Due &&&
  Apr 13  (Fri): & Exam 3 \\
Feb 28 (Wed): & 2nd Drop Deadline &&&
  Apr 27  (Fri): & Project 3 Due \\
Mar 9 (Fri): & Exam 2 &&&
   May 8 (Tues): & Final Exam
\end{tabular}
\end{itemize}

\item[] {\bf Students with Disability:} If you qualify for accommodations because of a disability, please submit to me a letter from Disability Services in a timely manner so that your needs may be addressed.  Disability Services determines accommodations based on documented disabilities.  Contact: 303-492-8671, Willard 322, or\\
\smallskip
	\centerline{\url{http://www.Colorado.EDU/disabilityservices}.}  
Disability Services' letters for students with disabilities indicate legally mandated reasonable accommodations.  The syllabus statements and answers to Frequently Asked Questions can be found at\\
\smallskip
	\centerline{\url{http://www.colorado.edu/disabilityservices}.}

\item[] {\bf Religious Obligations:} Campus policy regarding religious observances requires that faculty make every effort to reasonably and fairly deal with all students who, because of religious obligations, have conflicts with scheduled exams, assignments or required attendance.  In this class, you should give me at least a weeks notice if you will be missing an exam or a project due date due to a religious obligation. See full details at\\
\smallskip
	 \centerline{\url{http://www.colorado.edu/policies/fac_relig.html}.}

\item[] {\bf Classroom Behavior:} Students and faculty each have responsibility for maintaining an appropriate learning environment. Students who fail to adhere to such behavioral standards may be subject to discipline. Faculty have the professional responsibility to treat all students with understanding, dignity and respect, to guide classroom discussion and to set reasonable limits on the manner in which they and their students express opinions.  For more informations go to\\
\smallskip
	\centerline{\url{http://www.colorado.edu/policies/classbehavior.html}.}

\item[] {\bf Honor Code:} All students of the University of Colorado at Boulder are responsible for knowing and adhering to the academic integrity policy of this institution. Violations of this policy may include: cheating, plagiarism, aid of academic dishonesty, fabrication, lying, bribery, and threatening behavior.  All incidents of academic misconduct shall be reported to the Honor Code Council (303-725-2273). Students who are found to be in violation of the academic integrity policy will be subject to both academic sanctions from the faculty member and non-academic sanctions (including but not limited to university probation, suspension, or expulsion). Other information on the Honor Code can be found at\\
\smallskip
\centerline{\url{http://www.colorado.edu/policies/honor.html}}
and\\
\centerline{\url{http://www.colorado.edu/academics/honorcode/}.}

\item[] {\bf Closing Remarks:}  When does the learning happen?  It might happen in class, but most likely it happens when you sit down to do your homework.  Most of you can follow what I do on the board, but the question is, can you do it on your own?  To learn best, you must struggle with mathematics on your own.  It is supposed to be difficult (if it is not difficult for you, then I will gladly find things to challenge you).  However, if you are struggling too much, then there are resources available for you.  I am always happy to help you.  I want to help you.  If my office hours don't work for you, then we can probably find another time to meet.  You can also get help from each other.  Get a study buddy!  Help each other learn.  It is your responsibility to be aware of how well you understand the material.  Don't wait until it is too late if you need help. {\it Ask questions!} 

\end{itemize}

\end{document}
