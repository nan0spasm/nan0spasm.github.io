\documentclass[12pt]{article}
\usepackage{url}
\addtolength{\oddsidemargin}{-0.9in} 
\addtolength{\textwidth}{1.5in} 
\addtolength{\textheight}{2in} 
\addtolength{\topmargin}{-1.1in} 
\pagestyle{empty}
\begin{document} 

\centerline{{\large {\bf Math 2300: Analytic Geometry and Calculus II, Fall 2005}}} 
\centerline{{\large {\bf Section 001, MTWThF, 9:00--9:50 a.m., MUEN E126}}} 

\vskip .75cm
 
\noindent {{\large {\bf General Information:}}\\

\begin{tabular}{l l}
{\bf  Instructor:}  Dana Ernst\\
{\bf  Lectures:}  MTWF 9:00--9:50 a.m., MUEN E126\\
{\bf  Office:} Math 214\\
{\bf  Office Hours:} MWF 12:00--1:00 p.m (or by appointment)\\
{\bf  Email:} Dana.Ernst@colorado.edu\\
{\bf  Webpage:}  \url{http://math.colorado.edu/~ ernstd}\\
\\
{\bf  Teaching Assistant:}  Jonas D'Andrea\\
{\bf  Recitations:} Th 9:00--9:50 a.m., MUEN E126\\
{\bf  Office:} Math 340\\
{\bf  Email:} dandrea@colorado.edu\\
\end{tabular}\\

\vskip 10pt
\noindent{{\large{\bf Text:}}

\begin{itemize}
\item[] {\it Calculus (Early Transcendentals Version)}, sixth edition, by Edwards and Penney.
\end{itemize}

\vskip 10pt
\noindent {\large{\bf Prerequisites:}}

\begin{itemize}
\item[] Math 1300 or 1310 (or equivalent), with a grade of C or higher.
\end{itemize}

\vskip 10pt
\noindent {\large{\bf About the Course:}}

\begin{itemize}
\item[] As you know, calculus is a study of functions. The main tools
are differentiation, which measures instantaneous change in a function,
and integration, which gauges the cumulative effect of that 
change. The crowning achievement of first semester calculus is the fundamental
theorem, which explains how differentiation and integration are related.
\smallskip
\item[] In this second semester course, we focus on the important problem of how
one actually integrates functions, and will learn techniques that are exact
(for deriving formulas in science and engineering), and approximation techniques,
which are important in practice.
\smallskip
\item[] Among the techniques of integration we will learn are
the methods of partial fractions and integration by parts, and the
method of substitution (in essence, changing variables). We will generalize
the notion of changing variables to study parameterization and new
coordinate systems, like polar coordinates.
Along the way we will study more about logarithms and hyperbolic
trigonometric functions, which are useful in their own right and
for integration. We will also study the related notion of conic sections.
\smallskip
\item[] Perhaps the main topic of the semester
will be a study of infinite series,
which will let us consider ``nice'' functions as ``infinite polynomials,'' called Taylor Series,
which makes them much easier to do calculus with. Taylor Series will also allow us to
integrate functions when our previous techniques do not suffice, and provide useful approximations
of functions for numerical techniques.
\end{itemize}

\vskip 10pt
\noindent {\large{\bf Class Meetings:}}

\begin{itemize}
\item[] This course will meet five days a week.  On MTWF, class will be
conducted by your instructor, while each Thursday, your TA
will lead a recitation.
\end{itemize}

\vskip 10pt
\noindent {\large{\bf Requirements and Grading of the Course:}}

\begin{itemize}
\item[] {\bf Homework assignments} through the first
exam appear at the back of this handout.  (Homework for the rest of
the semester will be assigned at later dates.)  Note that assignments
will generally be collected on each Tuesday, Wednesday, and Friday.  
Homework will be returned on Thursdays during recitation. No late
homework will be accepted.  Homeworks must be legible and stapled
together. Every question on the homework
must be answered with a complete sentence: you must have a subject and
a verb. The subject and verb can be written in mathematical notation.
For example, if the question is, ``what is $y$?'' You may answer in the
complete sentence,
``$y=x^2+3$,'' but not with the fragment, ``$x^2+3.$''
Homeworks will be graded out of 10 points. You will
get 4 points for doing all the homework, and then three problems
will be selected to be graded, and each will be worth 2 points.
Your total homework score will be rescaled to contribute
{\bf 100} points to your final grade.
\smallskip
\item[] {\bf Quizzes:} In a typical week (all except the first week, and the weeks in
which there are exams), there will be a 10-15 minute quiz. Typically the
quizzes will be on Tuesday, but in weeks where we need more time to cover
new material, quizzes will be on Thursday. Your aggregate
quiz scores will contribute {\bf 100 points} to your final grade.
\smallskip
\item[] {\bf Exams:}  There will be three  midterm exams, given at {\bf 5:15--6:45 p.m.} on:

\vskip 2pt
\smallskip
\centerline {\bf Wednesday, 9/14; Wednesday, 10/5; Wednesday, 11/9;}
\vskip 2pt

respectively.  (These times and dates are fixed in stone, so make sure {\it now} that you have no conflicting plans.)  The exams will be in HUMN 1B50.  Each exam will be worth {\bf 100 points} of your final grade. The hour exams are not cumulative.
\smallskip
\item[] The {\bf Final Exam}, scheduled for {\bf Wednesday, DECEMBER 14, from 4:30--7:00 p.m.}, in a room to be announced, will constitute the remaining {\bf 200 points} of your final grade.  The final will, unlike hour exams, be {\it cumulative}.
\end{itemize}

\vskip 10pt
\noindent {\large{\bf The Calculus Help Lab and Tutoring:}}

\begin{itemize}
\item[] You may seek assistance at the Calculus 2 Help Lab, which will run from 3:30--5:30 p.m., Mondays in ECCR 155 and Tuesdays in MUEN 118. The lab will start during the second week of classes. Online tutoring is available at \url{http://onlinetutor.colorado.edu}. The Math Department maintains a list of qualified tutors in Math 260 (rates and availability vary).
\end{itemize}

\vskip 10pt
\noindent {\large{\bf Regarding Calculators:}}

\begin{itemize}
\item[] Calculators are not allowed on this quizzes or the exams. On the other hand, the quizzes and exams will be designed so that you will not need them.
\end{itemize}

\vskip 10pt
\noindent {\large{\bf Additional Information:}}

\begin{itemize}
\item[] The last day to drop the course without fee or a ``W'' on your
transcript is {\bf September 7}.  The last day to drop a course without
petitioning the Dean is {\bf October 5}. Labor Day Holiday is Monday, September 5,
and Fall Break is October 13 and 14. Thanksgiving Holiday is November 24 and 25. Note that
Tuesday, November 22 is considered as a Thursday by the University, so you should
go to recitation that day!
\smallskip
\item[] Please inform your instructor as soon as possible
should you need, due to your observance
of a religious holiday, to miss an exam, quiz, or homework. It is our
policy to be as accommodating as practicable in these circumstances.
\smallskip
\item[] If you qualify for accommodations because of a disability, please submit to
your instructor a letter from Disability Services in a timely manner so that your needs may
be addressed.  Disability Services determines accommodations based on
documented disabilities.  Contact:
\vskip 2pt
\smallskip
\centerline {303-492-8671, Willard 322, \url{http://www.Colorado.EDU/disabilityservices}.}
\vskip 2pt
\smallskip
\item[] All students of the University of Colorado at Boulder are responsible for
knowing and adhering to the academic integrity policy of this institution.
Violations of this policy may include: cheating, plagiarism, aid of academic
dishonesty, fabrication, lying, bribery, and threatening behavior.  All
incidents of academic misconduct shall be reported to the Honor Code Council
(honor@colorado.edu; 303-725-2273). Students who are found to be in violation
of the academic integrity policy will be subject to both academic sanctions
from the faculty member and non-academic sanctions (including but not limited
to university probation, suspension, or expulsion). Other information on the
Honor Code can be found at
\vskip 2pt
\smallskip
\centerline {\url{http://www.colorado.edu/policies/honor.html}}
\vskip 2pt
and at
\vskip 2pt
\centerline {\url{http://www.colorado.edu/academics/honorcode}.}
\vskip 2pt
\smallskip

\item[] The University of Colorado at Boulder policy on Discrimination and Harassment
\\(\url{http://www.colorado.edu/policies/discrimination.html}), the University of
Colorado policy on Sexual Harassment and the University of Colorado policy on
Amorous Relationships applies to all students, staff and faculty.  Any student,
staff or faculty member who believes s/he has been the subject of
discrimination or harassment based upon race, color, national origin, sex, age,
disability, religion, sexual orientation, or veteran status should contact the
Office of Discrimination and Harassment (ODH) at 303-492-2127 or the Office of
Judicial Affairs at 303-492-5550.  Information about the ODH and the campus
resources available to assist individuals regarding discrimination or
harassment can be obtained at \url{ http://www.colorado.edu/odh}.
\end{itemize}
\vskip 10pt

\noindent {\large {\bf Closing Remarks:}}
\begin{itemize}
\item[] When does the learning happen?  It might happen in class, but most likely it happens when you sit down to do your homework.  Most of you can follow what I do on the board, but the question is, can you do it on your own?  To learn best, you must struggle with mathematics on your own.  It is supposed to be difficult (if it is not difficult for you, then I will gladly find things to challenge you).  However, if you are struggling too much, then there are resources for you.  I am always happy to help you.  I want to help you.  If my office hours don't work for you, then we can probably find another time to meet.  You can also get help from each other.  Get a study buddy!  Help each other learn.  In addition, you can get help in Calculus Help Lab (MT, 3:30--5:30 p.m., location TBA). If you are having difficulty, then get some help.  It is your responsibility to be aware of how well you understand the material.  There are many resources available to you; use them!  It also helps to learn how to read the textbook.  Regardless of how difficult this may seem at times, it will help you.  Try reading the textbook before and after class.
\end{itemize}

                                                                             

\end{document}
