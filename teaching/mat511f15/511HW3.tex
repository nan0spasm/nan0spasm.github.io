\documentclass[11pt]{scrartcl}
\usepackage[scale=1.5]{ccicons}
\usepackage[notextcomp]{kpfonts} 
\usepackage[margin=1in]{geometry}
\usepackage{amsthm,amssymb}
\usepackage{graphicx}
\usepackage{enumitem}
\usepackage{bm}
\usepackage{tabu}
\usepackage{tikz}

\usepackage{color}
\definecolor{darkblue}{rgb}{0, 0, .6}
\definecolor{grey}{rgb}{.7, .7, .7}
\usepackage[breaklinks]{hyperref}
\hypersetup{
	colorlinks=true,
	linkcolor=darkblue,
	anchorcolor=darkblue,
	citecolor=darkblue,
	pagecolor=darkblue,
	urlcolor=darkblue,
	pdftitle={},
	pdfauthor={}
}

\usepackage{fancyhdr}
\pagestyle{fancy}
\lhead{MAT 511 - Fall 2015}
\chead{}
\rhead{Due Wednesday, September 16}
%\lfoot{}%\scriptsize This work is licensed under the \href{http://creativecommons.org/licenses/by-sa/3.0/us/}{Creative Commons Attribution-Share Alike 3.0 License}.} 
%\cfoot{}
%\rfoot{\ccbysa}
\renewcommand{\headrulewidth}{.4pt}
%\renewcommand{\footrulewidth}{.4pt}

\theoremstyle{definition}
\newtheorem{theorem}{Theorem}
\newtheorem{acknowledgement}[theorem]{Acknowledgement}
\newtheorem{algorithm}[theorem]{Algorithm}
\newtheorem{axiom}[theorem]{Axiom}
\newtheorem{case}[theorem]{Case}
\newtheorem{claim}[theorem]{Claim}
\newtheorem*{claim*}{Claim}
\newtheorem{conclusion}[theorem]{Conclusion}
\newtheorem{condition}[theorem]{Condition}
\newtheorem{conjecture}[theorem]{Conjecture}
\newtheorem{corollary}[theorem]{Corollary}
\newtheorem{criterion}[theorem]{Criterion}
\newtheorem{definition}[theorem]{Definition}
\newtheorem{example}[theorem]{Example}
\newtheorem{exercise}[theorem]{Exercise}
\newtheorem{journal}[theorem]{Journal}
\newtheorem{lemma}[theorem]{Lemma}
\newtheorem{notation}[theorem]{Notation}
\newtheorem{problem}[theorem]{Problem}
\newtheorem{proposition}[theorem]{Proposition}
\newtheorem{remark}[theorem]{Remark}
\newtheorem{solution}[theorem]{Solution}
\newtheorem{summary}[theorem]{Summary}
\newtheorem{skeleton}[theorem]{Skeleton Proof}
\newtheorem{activity}[theorem]{Activity}
\newtheorem{intuitivedef}[theorem]{Intuitive Definition}

\newcommand{\blankline}{\pagebreak[2]\vspace{.5\baselineskip}}

\setlength{\parindent}{0pt}

%Useful for cut and paste
%\begin{enumerate}[label=\rm{(\alph*)}]

\begin{document}

\title{Homework 3}
\subtitle{Abstract Algebra I}
\date{}

\maketitle
\thispagestyle{fancy}

Complete the following problems. Assume that the dihedral group of order $2n$ is defined by $D_{2n}=\langle r,s\mid r^n=s^2=1, rs=sr^{-1}\rangle$.

\begin{problem}
Consider the dihedral group $D_{10}$.
\begin{enumerate}[label=\rm{(\alph*)}]
\item Draw the Cayley diagram for $D_{10}$ using the generators $r$ and $s$.  Be sure to label the vertices as words in $r$ and $s$ and use two different colors for the edges.
\item Find the order of each of the elements in $D_{10}$.
\end{enumerate}
\end{problem}

\begin{problem}
Consider the dihedral group $D_{2n}$.
\begin{enumerate}[label=\rm{(\alph*)}]
\item Use the generators and relations of $D_{2n}$ to show that every element of $D_{2n}$ that is not a power of $r$ has order 2. 
\item Prove that $D_{2n}$ is generated by $s$ and $sr$.
\item Prove that $\langle a,b\mid a^2=b^2=(ab)^n=1\rangle$ gives a presentation for $D_{2n}$ in terms of the generators $a=s$ and $b=sr$. \emph{Hint:} Check that each set of relations can be built from the other.
\item Draw the Cayley diagram for $D_{10}$ using the generators $s$ and $sr$.  Be sure to label the vertices and use two different colors for the edges.
\end{enumerate}
\end{problem}

\begin{problem}
Consider the dihedral group $D_{2n}$.
\begin{enumerate}[label=\rm{(\alph*)}]
\item If $n$ is odd and $n\geq 3$, prove that the identity is the only element of $D_{2n}$ that commutes with all the elements of $D_{2n}$.
\item What happens if $n$ is even and at least 4?  That is, if $n$ is even and $n\geq 4$, are there any non-identity elements in $D_{2n}$ that commute with all of the elements of $D_{2n}$?  You do not need to prove your claim.
\end{enumerate}
\end{problem}

\begin{problem}
Let $Y=\langle u,v\mid u^4=v^3=1,uv=v^2u^2\rangle$.
\begin{enumerate}[label=\rm{(\alph*)}]
\item Prove that $v^2=v^{-1}$.
\item Prove that $v$ commutes with $u^3$. \emph{Hint:} Try messing with $(v^2u^2)(uv)$.
\item Prove that $v$ commutes with $u$.  \emph{Hint:} First produce $u^9=u$ and then use a previous part.
\item Prove that $uv=1$.
\item Prove that $u=1=v$ and deduce that $Y=\{1\}$.  \emph{Hint:} Do something clever with $u^4v^3=1$.
\end{enumerate}
The upshot of this problem is that sometimes innocent-looking presentations can collapse to the trivial group.
\end{problem}

\begin{problem}
Let $G$ be the group of rigid motions in $\mathbb{R}^3$ of a cube.  Prove that $|G|=24$. \emph{Hint:} One approach is to find the number of positions to which an adjacent pair of vertices can be sent.
\end{problem}

\begin{problem}
Consider the symmetric group $S_3$.
\begin{enumerate}[label=\rm{(\alph*)}]
\item Explicitly show that the adjacent $2$-cycles $(1,2)$ and $(2,3)$ generate $S_3$.
\item Draw the Cayley diagram for $S_3$ using $(1,2)$ and $(2,3)$ as generators. Be sure to label the vertices and use two different colors for the edges.
\end{enumerate}
\end{problem}

\begin{problem}
Find the order of $(1,12,8,10,4)(2,13)(5,11,7)(6,9)$ in $S_{13}$.
\end{problem}

\begin{problem}
Prove that if $\sigma=(a_1,a_2,\ldots,a_m)\in S_n$, then $|\sigma|=m$.
\end{problem}

\begin{problem}
Prove that the order of an element in $S_n$ equals the least common multiple of the lengths of the cycles in its cycle decomposition.
\end{problem}

\begin{problem}
We define the \textbf{quaternion group} to be the group $Q_8=\{1,-1,i,-i,j,-j,k,-k\}$ having the Cayley diagram with generators $i, j, -1$ given below.  In this case, 1 is the identity.

\tikzstyle{vert} = [circle, draw, fill=grey,inner sep=0pt, minimum size=6.5mm]
\tikzstyle{b} = [draw,very thick,blue,-stealth]
\tikzstyle{r} = [draw, very thick, red,-stealth]
\tikzstyle{g} = [draw, very thick, green, stealth-stealth]

\begin{center}
\begin{tikzpicture}[scale=1.5,auto]
\node (1) at (135:2) [vert] {\scriptsize $1$};
\node (i) at (45:2) [vert] {\scriptsize $i$};
\node (k) at (-45:2) [vert] {\scriptsize $k$};
\node (j) at (-135:2) [vert] {\scriptsize $j$};
\node (-1) at (135:1) [vert] {\scriptsize $-1$};
\node (-i) at (45:1) [vert] {\scriptsize $-i$};
\node (-k) at (-45:1) [vert] {\scriptsize $-k$};
\node (-j) at (-135:1) [vert] {\scriptsize $-j$};

\path[b] (1) to (i);
\path[b] (i) to (-1);
\path[b] (-1) to (-i);
\path[b] (-i) to (1);

\path[b] (-j) to (-k);
\path[b] (-k) to (j);
\path[b] (j) to (k);
\path[b] (k) to (-j);

\path[r] (-k) to (-i);
\path[r] (-i) to (k);
\path[r] (k) to (i);
\path[r] (i) to (-k);

\path[r] (1) to (j);
\path[r] (j) to (-1);
\path[r] (-1) to (-j);
\path[r] (-j) to (1);

\path[g] (1) to (-1);
\path[g] (j) to (-j);
\path[g] (i) to (-i);
\path[g] (k) to (-k);

\end{tikzpicture}
\end{center}
\begin{enumerate}[label=\rm{(\alph*)}]
\item Create the group table for $Q_8$.
\item Find the orders of each of the elements in $Q_8$.
\item The Cayley diagram given above was created using a generating set of size 3.  Can you identify a smaller generating set?
\end{enumerate}
\end{problem}

\end{document}