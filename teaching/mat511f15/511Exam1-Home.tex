\documentclass[11pt]{scrartcl}

\usepackage[scale=1.5]{ccicons}
\usepackage[notextcomp]{kpfonts}
\usepackage{multicol}
\usepackage{url}
\usepackage{array}
\usepackage{multicol}
\usepackage{tabu}
\usepackage{tikz}
\usetikzlibrary{shapes.geometric}
\usepackage{fancyhdr}
\usepackage[margin=1in]{geometry}
\usepackage[hang,flushmargin,symbol*]{footmisc}
\usepackage{amsmath}
\usepackage{amsthm}
\usepackage{amssymb}
\usepackage{mathtools}
\usepackage{enumitem}
\usepackage{graphicx}
\usepackage{color}
\definecolor{darkblue}{rgb}{0, 0, .6}
\definecolor{grey}{rgb}{.7, .7, .7}
\usepackage[breaklinks]{hyperref}

\theoremstyle{definition} 
\newtheorem{theorem}{Theorem}
\newtheorem{lemma}[theorem]{Lemma}
\newtheorem{claim}[theorem]{Claim}
\newtheorem{corollary}[theorem]{Corollary}
\newtheorem{conjecture}[theorem]{Conjecture}
\newtheorem{definition}[theorem]{Definition}
\newtheorem{example}[theorem]{Example}
\newtheorem{remark}[theorem]{Remark}
\newtheorem{important}[theorem]{Important Note}
\newtheorem{recall}[theorem]{Recall}
\newtheorem{note}[theorem]{Note}
\newtheorem{question}[theorem]{Question}
\newtheorem*{definition*}{Definition}

\newcommand{\ds}{\displaystyle}
\newcommand{\lcm}{\operatorname{lcm}}
\newcommand{\Rng}{\operatorname{Rng}}

\setlength{\parindent}{0pt}
\setlength{\fboxsep}{10pt}

%%%%%%Header/Footer%%%%%%%

\pagestyle{fancy}

\lhead{MAT 511 - Fall 2015}
\chead{}
\rhead{Exam 1 (Take-home portion)}
\lfoot{\scriptsize This work is licensed under the \href{https://creativecommons.org/licenses/by-sa/4.0/}{Creative Commons Attribution-Share Alike 4.0 License}.} 
\cfoot{}
\rfoot{\ccbysa}
\renewcommand{\headrulewidth}{.4pt}
\renewcommand{\footrulewidth}{.4pt}

%%%%%%%%%%%%%%%%%%%

\begin{document}

\begin{center}

  \fbox{\parbox{6in}{
    \vspace{5pt}
    \textbf{\large Your Name:}
    \vspace{5pt}
  }}
  
  \bigskip
  
  \fbox{\parbox{6in}{
    \vspace{5pt}
    \textbf{\large Names of Any Collaborators:}
    \vspace{5pt}
  }}

\end{center}

\section*{Instructions}

This portion of Exam 1 is worth a total of 20 points and is due at the beginning of class on \textbf{Monday, October 19}.  Your total combined score on the in-class portion and take-home portion is worth 20\% of your overall grade.  

\bigskip

I expect your solutions to be \emph{well-written, neat, and organized}.  Do not turn in rough drafts.  What you turn in should be the ``polished'' version of potentially several drafts.  
 
\bigskip

Feel free to type up your final version.  The \LaTeX\ source file of this exam is also available if you are interested in typing up your solutions using \LaTeX.  I'll gladly help you do this if you'd like.

\bigskip

The simple rules for the exam are:

\begin{enumerate}
\item You may freely use any theorems that we have discussed in class, but you should make it clear where you are using a previous result and which result you are using.  For example, if a sentence in your proof follows from Theorem xyz, then you should say so. 
\item Unless you prove them, you cannot use any results that we have not yet covered.
\item You are \textbf{NOT} allowed to consult external sources when working on the exam.  This includes people outside of the class, other textbooks, and online resources.
\item You are \textbf{NOT} allowed to copy someone else's work.
\item You are \textbf{NOT} allowed to let someone else copy your work.
\item You are allowed to discuss the problems with each other and critique each other's work.
\end{enumerate}

\begin{center}
\textbf{I will vigorously pursue anyone suspected of breaking these rules.}
\end{center}

\bigskip

You should \textbf{turn in this cover page} and all of the work that you have decided to submit. \textbf{Please write your solutions and proofs on your own paper.}

\bigskip

To convince me that you have read and understand the instructions, sign in the box below.

\bigskip

  \fbox{\parbox{6in}{
    \vspace{5pt}
    \textbf{\large Signature:} \hfill
    \vspace{5pt}
  }}

\bigskip

Good luck and have fun!

\newpage

Complete the following tasks.  Write your solutions on your own paper and please put the problems in order.

\begin{enumerate}

\item (4 points) For Problem 7 on the in-class portion of the exam, you proved one of the following.  Now, prove the one you didn't prove.
\begin{enumerate}[label=\text{(\alph*)}]
\item Prove that the group of rigid motions of a tetrahedron is isomorphic to a subgroup of $S_4$.
\item Suppose $\phi:G\to H$ is a group homomorphism.  We know that $\ker(\phi)\leq G$.  Prove that $N_G(\ker(\phi))=G$.
\end{enumerate}

\item (4 points) For Problem 9 on the in-class portion of the exam, you proved one of the following.  Now, prove the one you didn't prove.
\begin{enumerate}[label=\text{(\alph*)}]
\item Let $G$ be a group and define $\phi:G\to G$ via $\phi(g)=g^2$ for all $g\in G$.  Prove that $\phi$ is a homomorphism iff $G$ is abelian.\footnote{In case it wasn't obvious, $|z|$ is referring to the order of $z$ in the group, not its length.  More evidence that we use the absolute value delimiters for too many things in mathematics?}
\item Define the group $G=\{z\in\mathbb{C}\mid |z|<\infty\}$ where the operation is ordinary multiplication of complex numbers.\footnote{You do not need to prove that this is a group.  It turns out that $G$ is the torsion subgroup of $\mathbb{C}\setminus\{0\}$.} Prove that the function $\psi:G\to G$ defined via $\psi(z)=z^k$ is a surjective homomorphism but not an isomorphism, where $k$ is a fixed positive integer greater than 1.
\end{enumerate}

Some of problems below require the following definition.

\bigskip

\begin{definition*}
A subgroup $H$ of a group $G$ is called \textbf{normal} if $gHg^{-1}=H$ for all $g\in G$.  Equivalently, $H$ is a normal subgroup of $G$ provided $N_G(H)=G$.  If $H$ is normal in $G$, then we write $H\unlhd G$.
\end{definition*}

\item (4 points each) Prove \textbf{three} of the following. Make sure that you are only using results we have developed so far this semester.

\begin{enumerate}[label=\text{(\alph*)}]
\item If $G$ is a finite group such that $G=\langle x,y\rangle$, where $x\neq y$, $|x|=|y|=2$, and $|xy|=n$, then prove that $G\cong D_{2n}$. 
\item If $G$ is a finite group and $H$ is a subgroup of $G$ such that $H=\langle S\rangle$ for some $S\subseteq G$, then prove that $H\unlhd G$ iff $gSg^{-1}\subseteq H$ for all $g\in G$.
\item If $H$ is a subgroup of $G$, then prove that $H\unlhd G$ iff $gH=Hg$ for all $g\in G$.\footnote{Recall that $gH=\{gh\mid h\in H\}$ and similar for $Hg$.}
\item If $H\leq G$, then prove that $N_G(H)$ is the largest subgroup of $G$ in which $H$ is normal.\footnote{Note that step one is prove that $H$ is actually normal in $N_G(H)$.}
\item If $G$ is a finite group and $H\leq G$ such that $|G|=2|H|$, then prove that $H\unlhd G$.
\end{enumerate}

\item (2 points) \textbf{Optional Bonus Question!} In Problem 3(b) we required that $G$ be finite.  Is the result still true if $G$ is not finite?  Justify your answer.

\end{enumerate}

\end{document}