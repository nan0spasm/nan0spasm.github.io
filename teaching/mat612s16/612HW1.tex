\documentclass[11pt]{scrartcl}
\usepackage[scale=1.5]{ccicons}
\usepackage[notextcomp]{kpfonts} 
\usepackage[margin=1in]{geometry}
\usepackage{amsthm,amssymb}
\usepackage{graphicx}
\usepackage{enumitem}
\usepackage{bm}
\usepackage{tabu}
\usepackage{tikz}

\usepackage{color}
\definecolor{darkblue}{rgb}{0, 0, .6}
\definecolor{grey}{rgb}{.7, .7, .7}
\usepackage[breaklinks]{hyperref}
\hypersetup{
	colorlinks=true,
	linkcolor=darkblue,
	anchorcolor=darkblue,
	citecolor=darkblue,
	pagecolor=darkblue,
	urlcolor=darkblue,
	pdftitle={},
	pdfauthor={}
}

\usepackage{fancyhdr}
\pagestyle{fancy}
\lhead{MAT 612 - Spring 2016}
\chead{}
\rhead{Due Wednesday, January 27}
\renewcommand{\headrulewidth}{.4pt}

\theoremstyle{definition}
\newtheorem{theorem}{Theorem}
\newtheorem{acknowledgement}[theorem]{Acknowledgement}
\newtheorem{algorithm}[theorem]{Algorithm}
\newtheorem{axiom}[theorem]{Axiom}
\newtheorem{case}[theorem]{Case}
\newtheorem{claim}[theorem]{Claim}
\newtheorem*{claim*}{Claim}
\newtheorem{conclusion}[theorem]{Conclusion}
\newtheorem{condition}[theorem]{Condition}
\newtheorem{conjecture}[theorem]{Conjecture}
\newtheorem{corollary}[theorem]{Corollary}
\newtheorem{criterion}[theorem]{Criterion}
\newtheorem{definition}[theorem]{Definition}
\newtheorem{example}[theorem]{Example}
\newtheorem{exercise}[theorem]{Exercise}
\newtheorem{journal}[theorem]{Journal}
\newtheorem{lemma}[theorem]{Lemma}
\newtheorem{notation}[theorem]{Notation}
\newtheorem{problem}[theorem]{Problem}
\newtheorem{proposition}[theorem]{Proposition}
\newtheorem{remark}[theorem]{Remark}
\newtheorem{solution}[theorem]{Solution}
\newtheorem{summary}[theorem]{Summary}
\newtheorem{skeleton}[theorem]{Skeleton Proof}
\newtheorem{activity}[theorem]{Activity}
\newtheorem{intuitivedef}[theorem]{Intuitive Definition}

\DeclareMathOperator{\Aut}{Aut}
\DeclareMathOperator{\Inn}{Inn}
\DeclareMathOperator{\Stab}{Stab}

\newcommand{\blankline}{\pagebreak[2]\vspace{.5\baselineskip}}

\setlength{\parindent}{0pt}

%Useful for cut and paste
%\begin{enumerate}[label=\rm{(\alph*)}]

\begin{document}

\title{Homework 1}
\subtitle{Abstract Algebra II}
\date{}

\maketitle
\thispagestyle{fancy}

Complete the following problems. Note that you should only use results that we've discussed so far this semester or last semester.

\begin{problem}
Determine whether each of the following statements is `True' or `False'.  If a statement is False, provide a counterexample.  If a statement is True, then either provide a reference or a proof.

\begin{enumerate}[label=\rm{(\alph*)}]
\item The image of a ring with 10 elements under some ring homomorphism may consist of 3 elements.
\item Every subring of an integral domain is an integral domain.
\item If $R$ is an integral domain but not a field, then it is possible that $R$ contains a subring that is a field.
\item Every ring consisting of exactly 5 elements is a field.
\item If $R$ is a finite commutative ring with 1 having no zero divisors, then $R$ is a field.
\item Suppose $R$ is an integral domain $R$. Then $R/I$ is a field iff $I$ is a prime ideal.
\item The set of matrices
\[
F=\left\{\begin{bmatrix}
1 & 0\\
0 & 1
\end{bmatrix},
\begin{bmatrix}
1 & 1\\
1 & 0
\end{bmatrix},
\begin{bmatrix}
0 & 1\\
1 & 1
\end{bmatrix},
\begin{bmatrix}
0 & 0\\
0 & 0
\end{bmatrix}
\right\}
\]
with entries from the field $\mathbb{Z}_2$ is a field, under ordinary matrix addition and multiplication.
\item Suppose $R$ is a commutative ring.  If $ra\in I$ for all $r\in R$ and $a\in I$, then $I$ is a two-sided ideal in $R$.
\end{enumerate}
\end{problem}

\begin{problem}
Let $R$ be a commutative ring with 1 and let $U(R)$ be the group of units in $R$.  Prove that $R$ has a unique maximal ideal iff $R\setminus U(R)$ is an ideal.  \emph{Note:} You may assume that maximal ideals exist.
\end{problem}

\begin{problem}
A \textbf{simple ring} is a ring with no nonzero proper 2-sided ideals.  If $R$ is a ring, then the \textbf{center} of $R$ is defined to be $Z(R):=\{x\in R\mid rx=xr\text{ for all } r\in R\}$.  Prove that the center of a simple ring with 1 is a field.  \emph{Note:} You must first show that the center is a subring.
\end{problem}

\begin{problem}
Assume $R$ is a commutative ring with 1. Prove that the ideal $(x)$ in $R[x]$ is a maximal ideal iff $R$ is a field.
\end{problem}

\begin{problem}
Assume $R$ is a commutative ring with $1\neq 0$ and for each $r\in R$, there exists an integer $n>1$ such that $r^n=r$. Prove that every prime ideal of $R$ is maximal.
\end{problem}

\begin{problem}
Assume $F$ is a finite field. Prove that there exists a prime $p$ such that all non-zero elements of $F$ have an additive order of $p$.
\end{problem}

\begin{problem}
Assume $R$ is a Euclidean Domain.  Let $m$ be the minimum integer in the set of norms of nonzero elements of $R$. Prove that every nonzero element of $R$ of norm $m$ is a unit.  Deduce that a nonzero element of norm zero is a unit.
\end{problem}

\begin{problem}
Assume $R$ is a Euclidean Domain.  Prove that if $(a,b)=1$ and $a$ divides $bc$, then $a$ divides $c$. More generally, prove that if $a$ divides $bc$ with nonzero $a$, then $a/(a,b)$ divides $c$. \emph{Note:} I don't care what order you do these in.  Certainly, the general statement handles the case when $(a,b)=1$. However, you may find it useful to do the special case first.
\end{problem}

\end{document}