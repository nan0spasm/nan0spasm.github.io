\documentclass[11pt]{scrartcl}
\usepackage[scale=1.5]{ccicons}
\usepackage[notextcomp]{kpfonts} 
\usepackage[margin=1in]{geometry}
\usepackage{amsthm,amssymb}
\usepackage{graphicx}
\usepackage{enumitem}
\usepackage{bm}
\usepackage{tabu}
\usepackage{tikz}

\usepackage{color}
\definecolor{darkblue}{rgb}{0, 0, .6}
\definecolor{grey}{rgb}{.7, .7, .7}
\usepackage[breaklinks]{hyperref}
\hypersetup{
	colorlinks=true,
	linkcolor=darkblue,
	anchorcolor=darkblue,
	citecolor=darkblue,
	pagecolor=darkblue,
	urlcolor=darkblue,
	pdftitle={},
	pdfauthor={}
}

\usepackage{fancyhdr}
\pagestyle{fancy}
\lhead{MAT 612 - Spring 2016}
\chead{}
\rhead{Due Wednesday, April 6}
\renewcommand{\headrulewidth}{.4pt}

\theoremstyle{definition}
\newtheorem{theorem}{Theorem}
\newtheorem{acknowledgement}[theorem]{Acknowledgement}
\newtheorem{algorithm}[theorem]{Algorithm}
\newtheorem{axiom}[theorem]{Axiom}
\newtheorem{case}[theorem]{Case}
\newtheorem{claim}[theorem]{Claim}
\newtheorem*{claim*}{Claim}
\newtheorem{conclusion}[theorem]{Conclusion}
\newtheorem{condition}[theorem]{Condition}
\newtheorem{conjecture}[theorem]{Conjecture}
\newtheorem{corollary}[theorem]{Corollary}
\newtheorem{criterion}[theorem]{Criterion}
\newtheorem{definition}[theorem]{Definition}
\newtheorem{example}[theorem]{Example}
\newtheorem{exercise}[theorem]{Exercise}
\newtheorem{journal}[theorem]{Journal}
\newtheorem{lemma}[theorem]{Lemma}
\newtheorem{notation}[theorem]{Notation}
\newtheorem{problem}[theorem]{Problem}
\newtheorem{proposition}[theorem]{Proposition}
\newtheorem{remark}[theorem]{Remark}
\newtheorem{solution}[theorem]{Solution}
\newtheorem{summary}[theorem]{Summary}
\newtheorem{skeleton}[theorem]{Skeleton Proof}
\newtheorem{activity}[theorem]{Activity}
\newtheorem{intuitivedef}[theorem]{Intuitive Definition}

\DeclareMathOperator{\Aut}{Aut}
\DeclareMathOperator{\Inn}{Inn}
\DeclareMathOperator{\Stab}{Stab}
\DeclareMathOperator{\Char}{Char}

\newcommand{\blankline}{\pagebreak[2]\vspace{.5\baselineskip}}

\setlength{\parindent}{0pt}

%Useful for cut and paste
%\begin{enumerate}[label=\rm{(\alph*)}]

\begin{document}

\title{Homework 8}
\subtitle{Abstract Algebra II}
\date{}

\maketitle
\thispagestyle{fancy}

Complete the following problems. Note that you should only use results that we've discussed so far this semester or last semester.

\begin{problem}
Let $K/F$ be a field extension, where $F$ is a perfect field. If $f(x) \in F[x]$ has no repeated irreducible factors, prove that $f(x)$ has no repeated irreducible factors over $K$.	
\end{problem}

\begin{problem}
Prove that there are only a finite number of roots of unity in any finite extension $K$ of $\mathbb{Q}$. \emph{Note:} You may use the fact that $[\mathbb{Q}(\zeta):\mathbb{Q}] = \phi(k)$, where $\zeta \in K$ is a primitive $k$th root of unity.
\end{problem}

\begin{problem}
Let $\tau:\mathbb{C}\to \mathbb{C}$ be defined via $\tau(a+bi)=a-bi$ (complex conjugation).
\begin{enumerate}[label=\rm{(\alph*)}]
\item Prove that $\tau$ is an automorphism of $\mathbb{C}$.
\item Determine the fixed field of $\tau$.
\end{enumerate}
\end{problem}

\begin{problem}
Prove that $\mathbb{Q}(\sqrt{2})$ and $\mathbb{Q}(\sqrt{3})$ are not isomorphic.	
\end{problem}

\begin{problem}
Determine the automorphisms of the extension $\mathbb{Q}(\sqrt[4]{2})/\mathbb{Q}(\sqrt{2})$ explicitly.
\end{problem}

\begin{problem}
Consider $\Aut(\mathbb{R}/\mathbb{Q})$.
\begin{enumerate}[label=\rm{(\alph*)}]
\item Prove that any $\sigma\in \Aut(\mathbb{R}/\mathbb{Q})$ takes squares to squares and takes positive reals to positive reals.  Conclude that $a<b$ imples $\sigma(a)<\sigma(b)$ for every $a,b\in\mathbb{R}$.
\item Prove that
\[
-\frac{1}{m}<a-b<\frac{1}{m}\text{ implies } -\frac{1}{m}<\sigma(a)-\sigma(b)<\frac{1}{m}
\]
for every positive integer $m$. Conclude that $\sigma$ is a continuous map on $\mathbb{R}$.
\item Prove that any continuous map on $\mathbb{R}$ that is the identity on $\mathbb{Q}$ is the identity map, hence $\Aut(\mathbb{R}/\mathbb{Q})$
\end{enumerate}
\end{problem}

\begin{problem}
Consider the polynomial $p(x)=x^4-5x^2+6\in\mathbb{Q}[x]$.  Determine $\Aut(K/\mathbb{Q})$, where $K$ is the splitting field of $p(x)$.
\end{problem}

\end{document}